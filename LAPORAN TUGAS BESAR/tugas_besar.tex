\documentclass[journal,transmag]{IEEEtran}
\IEEEoverridecommandlockouts

\usepackage{cite}
\usepackage{amsmath,amssymb,amsfonts}
\usepackage{algorithmic}
\usepackage{graphicx}
\usepackage{textcomp}
\usepackage{xcolor}
\def\BibTeX{{\rm B\kern-.05em{\sc i\kern-.025em b}\kern-.08em
    T\kern-.1667em\lower.7ex\hbox{E}\kern-.125emX}}

\renewcommand{\figurename}{Gambar}
\renewcommand{\tablename}{Tabel}
\renewcommand{\IEEEkeywordsname}{Keywords}
\usepackage{caption}
\usepackage{pgf}

\graphicspath{{./Gambar/}}

\begin{document}
    
\title{Mengirim Data Signal Strength Hasil Pemindaian Jaringan WiFi menggunakan NodeMCU V3 ke Telegram Dengan Channel BotFather}

\author{\IEEEauthorblockN{ %
Johanes Wilian Ang\IEEEauthorrefmark{1},
Erwin Erikson\IEEEauthorrefmark{2}, 
Johnny\IEEEauthorrefmark{3},
Andrian Syah\IEEEauthorrefmark{4}
}
\IEEEauthorblockA{
\textit{Faculty of Information Technology} \\
\textit{Institut Teknologi Batam}\\
Batam, Indonesia \\
Email:
\IEEEauthorrefmark{1}1822002@student.iteba.ac.id, 
\IEEEauthorrefmark{2}1822003@student.iteba.ac.id,
\IEEEauthorrefmark{3}1822004@student.iteba.ac.id,
\IEEEauthorrefmark{4}1922009@student.iteba.ac.id
}}

\maketitle

\begin{abstract}
    Dalam sistem komunikasi data berbasis \emph{wireless}, pemanfaatan \emph{WiFi} menjadi pilihan banyak pengguna karena keunggulan mobilitas dan kecepatan transfer data. Kualitas \emph{signal strength} sangat berpengaruh dalam layanan komunikasi data ini, sehingga kualitas \emph{signal strength} pada jaringan \emph{WiFi} perlu diketahui. Pemindaian jaringan \emph{WiFi} menggunakan perangkat \emph{NodeMCU V3} dapat mengukur \emph{signal strength} dari masing-masing jaringan, lalu data hasil pemindaian dan pengukuran \emph{signal strength} dikirim ke Telegram menggunakan \emph{channel} BotFather melalui \emph{webserver} agar dapat diketahui hasilnya.
\end{abstract}

\begin{IEEEkeywords}
    \emph{WiFi}, \emph{signal strength}, \emph{NodeMCU V3}, \emph{webserver}, \emph{BotFather}
\end{IEEEkeywords}

\section{Pendahuluan}

\bibliographystyle{IEEEtran}
\bibliography{referensi.bib}
\end{document}